\documentclass[11pt]{article}
\usepackage{preamble}

\begin{document}
\section*{Specific Aims}
Diffusion MRI (dMRI) is a powerful, non-invasive tool for characterizing
neurological tissue microstructure on a macroscopic scale, and is widely used in
both research and clinical settings. New methods of reconstructing orientation
distribution functions (ODFs) from dMRI data are rapidly being developed, each
allowing for the identification of distinct axon fiber populations within each
voxel. The 3D orientations of these fiber populations are passed into
tractography algorithms, which are used in connectomics research to estimate the
paths of individual nerve tracts throughout the brain.

Efforts to validate both ODF reconstruction methods and tractography results
have typically relied on serial optical histology. For ODF validation, a
computer vision technique called ``structure tensor analysis'' uses image
intensity gradients to estimate voxel-wise fiber orientations in the
high-resolution histology data. These orientation estimates are then binned
across regions of interest the size of a dMRI voxel in order to form
ground-truth ODFs, which are then compared to ODFs from the same regions in dMRI
data.  Tractography validation is generally performed using neural tracers
injected into a small number of seed sites before the sample is
sacrificed. Voxel-wise comparisons between dMRI tractography results and the
ground truth tracts identified from these tracers are then used to evaluate the
performance of various tractography algorithms.

Both of these validation efforts rely on the labor-intensive process of
physically sectioning, staining, and optically scanning hundreds of slices of
the tissue of interest. The slices are necessarily at least 10-20 times thicker
than the achievable in-plane resolution (\url{~}5000 nm vs. 250 nm), yielding
non-isotropic volumetric reconstructions; distortions introduced by sectioning
further limit the ability to align the slices and extract faithful information
on the 3D orientation of fiber populations.

In recent years, synchrotron micro computed tomography (microCT) has emerged as
an alternative to serial optical histology for use as a dMRI validation dataset, allowing
for isotropic, 3D imaging of whole mouse brain specimens at sub-micron
resolution. We propose to develop and optimize a pipeline to use microCT data to validate
and characterize dMRI ODF reconstruction and tractography algorithms. The
results of this work will address limitations of previous histology-based
studies, while generating a publicly available whole-brain tractography atlas
through a combination of white matter image segmentation and ``ground-truth''
tractograms generated from microCT ODFs. The specific aims are:

\noindent\textbf{Aim 1: Optimize microCT data acquisition to maximize axon fiber bundle contrast.}
Specimens are stained with uranyl acetate, osmium tetroxide and lead citrate
prior to microCT imaging. The concentrations and staining protocol for these
agents will be optimized along with the synchrotron beam energy to maximize the
contrast of axon fiber bundles in the microCT data. Additionally, we will
install and characterize a new detector system with a higher resolution and
wider field of view to simplify and improve the microCT image reconstruction,
which currently relies on a mosaic data-stitching method to image whole mouse
brains.

\noindent\textbf{Aim 2: Evaluate dMRI reconstruction methods using ground-truth
  microCT ODFs.}
We will compute ground-truth ODFs from the microCT data across a whole mouse
brain using structure tensor analysis. The ODFs will be compared to dMRI ODFs
calculated on the same specimen using a variety of reconstruction methods,
generating algorithm-specific 3D spatial maps of dMRI accuracy. We will
investigate patterns in the morphological features of failure regions in
an effort to better understand modeling errors in the dMRI reconstruction
methods. 

\noindent\textbf{Aim 3: Generate a ground-truth tractography atlas from microCT
  ODFs and segmented axon fiber bundles.} We will generate a 3D tractography atlas
by running dMRI tractography algorithms on the ground-truth ODFs from
microCT. This will yield the ``best-performance'' results of various algorithms;
having mitigated the effects of inaccurate dMRI ODFs, this approach will allow
us to understand the inherent limitations in estimating tracts from ODF data.
Tractogram results will also be validated against segmented axon fiber bundles
from the microCT intensity data for algorithm-specific characterization.

Upon completion, this project will generate a comprehensive validation map for
both dMRI reconstruction methods and tractography, exploiting the advantages of
a novel dataset uniquely tailored to this purpose. The ground-truth tractography
atlas will be made publically available as an unprecedentedly comprehensive tool
for studies into the future role of dMRI in connectomics research.
\end{document}